\documentclass[conference]{IEEEtran}
\IEEEoverridecommandlockouts
% The preceding line is only needed to identify funding in the first footnote. If that is unneeded, please comment it out.
\usepackage{cite}
\usepackage{amsmath,amssymb,amsfonts}
\usepackage{algorithmic}
\usepackage{graphicx}
\usepackage{textcomp}
\usepackage{xcolor}
\def\BibTeX{{\rm B\kern-.05em{\sc i\kern-.025em b}\kern-.08em
    T\kern-.1667em\lower.7ex\hbox{E}\kern-.125emX}}
\begin{document}

\title{Mapping Topologies and Services for L3VPN Automatic Provisioning}

\author{\IEEEauthorblockN{Samier Barguil}
\IEEEauthorblockA{\textit{Universidad Autonoma de Madrid}\\
Madrid, Spain\\
samier.barguil@estudiante.uam.es}\\
\and
\IEEEauthorblockN{Guillermo Pajares,Oscar Gonzalez de Dios \\
and Victor Lopez Alvarez}
\IEEEauthorblockA{\textit{Telefonica I+D}\\
Madrid, Spain}\\
\and
\IEEEauthorblockN{Ricard Vilalta}
\IEEEauthorblockA{\textit{Centre Tecnologic \\
de Telecomunicacions de Catalunya \\
(CTTC/CERCA) \\
Casteldefels, Spain}}\\
}

\maketitle

\begin{abstract}
Current approaches to automatically configure transport services have several problems.  In the first place, there is a gap of standard interfaces between the Operational Support Systems (OSS), and Network Management Systems (NMSs); the available interfaces are typically proprietary, which limits network programmability as the effort is put on integration. Second, the orchestration across different NMSs (e.g., IP/MPLS, microwave and Optical Transport) is complicated to achieve, because each NMS is a highly specialized element for the vendor. There are lacks in interoperability with other vendors’ elements, and, especially, on the NMS to NMS communication. Finally, there are missing aspects in the standardization of the interface for upper-layer applications or services. The set of previous problems prevent deploying services in a rapid and automated fashion in the telecom environment. 

Network operators need a carrier SDN solution that can be deployed in brownfield scenarios. In order to solve this, this work presents the mapping of network topologies and services to enable automatic L3VPN provisioning with standard interfaces. The paper demonstrates that such automation is possible with proposed draft standards from IETF and validates and verifies its implementation.
\end{abstract}

\begin{IEEEkeywords}
L3VPN, Topology, Service, mapping, provisioning
\end{IEEEkeywords}

\section{Introduction}
Transport networks are responsible for forwarding aggregated traffic demands from multiple end-users consuming different services among different cities, regions, or continents. Transport network architecture was conceived for static scenarios, where there were no massive changes in the traffic patterns. This paradigm allows configuring the network as a slow process, since there was no need for the operator to reconfigure the deployed resources dynamically. The pressure to optimize the network deployment to reduce the CAPEX and OPEX investment requires to change their operational model.

Software-Defined Networking (SDN) originally \cite{b1} came with the concept of the decoupling of control and forwarding planes in the network elements. A centralized controller with the complete network view does the control plane functions. It runs intelligent algorithms and applications (either as part of the controller or on top of it) and instructs the nodes accordingly. This original view, when ported to real carrier networks \cite{b2}, has been evolved towards an architecture where the centralized control assists the network elements on the forwarding tasks, providing a single and unified control environment for network operations. The network elements retain control capabilities but leveraging on the centralized controller for end-to-end and cross-layer actions through programmable interfaces. 

In this work, we present the automation of L3VPN services in a service provider environment. The integration of a hierarchical controller with a commercial SDN Domain controller using RESTCONF NBI interfaces is done. Also, the work implements two service models L3NM \cite{b3} and UNI-Topology \cite{b4} that are exposed between the controllers. 

This work has the following sections: Section II explains the functional architecture for a carrier-grade SDN solution. Section III explains the Network Service Models used for the L3VPN definition. Section IV explains the topology models, and Section V explains the experimental validation and the results obtained. Finally, Section VI concludes this paper.

\section{ARCHITECTURAL PROPOSAL}

This paper uses the iFUSION architecture, which decouples the evolution of both network and service layers \cite{b2}. iFusion includes two control layers: Software Defined Transport Network (SDTN) controller, with an E2E network view and SDN Controller (SDNc) for each technology IP, optical and microwave. The proposed model stands on the use of interfaces leveraging on the latest developments in IETF, ONF, and OpenConfig.

\section{LAYER 3 VPN PROVISIONING}
The Layer 3 Virtual Private Network (VPN) service defined in RFC 4364 provides a multipoint, routed service to the customer over an IP/MPLS core. 
L3NM \cite{b3} is a Service Network Model. The L3NM YANG model is exposed by network controllers to manage and control the VPN Service configuration. The model describes a VPN Service in the service provider network. It contains information of the service provider network such as NE-ID and Interfaces-ID and might include allocated resources such as the Route Targets and Route Distinguishers.

The L3NM model is Network Centric, it is focused on the service provider internal network parameters. The data model can be used to facilitate communication between the service orchestrator and the SDTN network controller. Also, the SDTN network controller can further include network specific information in the service request (i.e. transport LSP binding, Router Targets assignation, routing profiles or encryption). In addition, it facilitates multi-domain orchestration, where logical resources (such as route targets or route distinguishers) must be synchronized between domains.

\section{TOPOLOGY MODEL}

The proposed control architecture relies on providing different levels of abstraction for each control layer. Therefore, the needs in terms of topology and knowledge of the service provider network differ among components.

In the proposed hierarchy, the service orchestration layer does not need to know about the internals of the network. The authors have proposed a UNI topology model in \cite{b4}. The SDTN controller exposes the set of nodes. L3 VPN services can be requested adding as attachment points the nodes. For that aim, the UNI topology builds the network data model defined in the "ietf-network" module, augmenting the nodes with service-attachment points.

\section{EXPERIMENTAL VALIDATION}

The experimental set-up implements an IP scenario, where the data plane consists of two interconnected routers acting as PE devices. Every router has two cards of 10x10G interfaces but only two operative client-facing ports (1/1/c10/1 and 1/1/c11/2 in each router) can be used to connect customers. Each port has been configured to support Dot1Q ethernet encapsulation, so several services can be attached to the same physical port. The control plane is formed by a commercial IP SDNc connected to the routers. The control plane of the routers is in the same network of the IP SDNc controller and NETCONF/YANG is the communication protocol between those. The SDTN controller is developed by Telefonica. As the SDTN has an E2E view, all the services creation requests must be received by the SDTN, processed and forwarded using RESTCONF/YANG to the SDN controllers.

The first goal to be achieved was to retrieve the UNI-Topology. An attachment point must be selected of the service attachment points available on retrieved topology. After that an End-to-End L3VPN between the selected nodes must be established. The SDTN controller provisions the service using RESTCONF + IETF L3NM requests and these requests are sent to the IP SDNc, which handles the request.

The full E2E L3VPN service establishment process is shown in the Fig. 1. First, the figure shows the RESTCONF request of the L3VPN service creation, how this is received by SDTN Controller, passed to SDNc and accepted by both (step 1). Once the L3VPN service has been created, it is needed to create the desired IE Profiles and VPN Nodes for the service. Steps 2 and 3 of Fig. 1 show the traces of the creation process of one IE Profile and one VPN Node, respectively, for the L3VPN service in SDNc and SDTN Controller.

\section{CONCLUSIONS}

This work demonstrates for the first time two key aspects for network operators in practical scenarios: the mapping between topologies and L3 services, and,  the provisioning of L3 services with L3SM and L3NM interfaces. To do so, this work  implements for the first time L3NM and UNI topology IETF contributions. Finally, the work demonstrates that such automation with current standards and validates the implementation against a commercial SDN controller


\begin{figure}[htbp]
\centerline{\includegraphics{fig1.png}}
\caption{Example of a figure caption.}
\label{fig}
\end{figure}

\section*{Acknowledgment}

This work was supported partially by the European Commission H2020-ICT-2016-2 METRO-HAUL project (G. A. 761727) and Spanish AURORAS (RTI2018-099178).

\section*{References}

\begin{thebibliography}{00}
\bibitem{b1} D. Kreutz, et al., “Software-Defined Networking: A Comprehensive Survey”, Proc. of the IEEE, Vol. 103, Issue 1, pp. 14- 76, January 2015.
\bibitem{b2} L.M. Contreras, et al, “iFUSION: Standards-based SDN Architecture for Carrier Transport Network”, in Proc. of the IEEE CSCN, October 2019.
\bibitem{b3} Aguado, a., et al (2020). draft-ietf-opsawg-l3sm-l3nm-01 -A Layer 3 VPN Network YANG Model. [online] Tools.ietf.org. Available at: https://tools.ietf.org/html/draft-ietf-opsawg-l3sm-l3nm-01 [Accessed 2 Jan. 2020].
\bibitem{b4} Gonzalez de Dios, O., et al (2020). draft-ogondio-opsawg-uni-topology-00 - A YANG Model for User-Network Interface (UNI) Topologies. [online] Tools.ietf.org. Available at: https://tools.ietf.org/html/draft-ogondio-opsawg-uni-topology-00 [Accessed 2 Jan. 2020]
\end{thebibliography}

\end{document}
